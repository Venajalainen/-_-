\documentclass{report}

\usepackage[utf8]{inputenc}
\usepackage[a-1b]{pdfx}
\usepackage[russian]{babel}
\usepackage{pgfplots}
\usepackage{amsfonts}
\usepackage{mathtools}

\pgfplotsset{compat=1.18}

\title{Конспект Н.Г. де Брёйна Аисмптоические методы в анализе}
\author{Наумов А.Ю.}
\date{Лето 2024}



\begin{document}
\section{Асимптотика}
Часто случается, что нужно вычислить определенную величину. Однако её вычисление потребует чрезвычайно большое число дорогостоющих действий, что может привести к невозможности узнать нужную величину.
В таких случаях может понадобиться какой-то иной метод, позволяющий найти хотя бы достойное приближение. В подобном случае мы можем говорить об асимптотическом приближении.
Типичной асимптотической формулой является формула Стирлинга:
\begin{center}
\[ \lim\limits_{n \to \infty} \frac{n!}{e^{-n}n^{n}\sqrt{2\pi n}} = 1 \]
\end{center}
\begin{center}
    \begin{tikzpicture}[]
        \begin{axis}[
            title = Факториал и его приближение,
            xlabel = {$x$},
            ylabel = {$y$},
            xmin = 0, xmax = 10,
            ymin = 0,
            ]
            \addplot [
                black,
                domain = 0:7,
                samples = 200,
            ] {sqrt(2*pi*x)*(x/e)^x};
            \addplot[
                black,
                scatter,
                only marks,
            ] table {
                x   y
                0   1
                1   1
                2   2
                3   6
                4   24
                5   120
                6   720
                7   5040
            };
        \end{axis}
    \end{tikzpicture}
\end{center}
Формула Стирлинга даёт удобное приближение, и чем больше n, тем меньше относительная погрешность:

\begin{center}
\[ \delta = 100\% * \frac{f(x) - A}{A} \textrm{, где A-истинная виличина в точке x, $f(x)$-вычисленное значение} \]
\end{center}

Существует одна знаменитая асимптотическая формула, намного глубокая чем предыдущая. Для $\forall x>0$ обозначим через
$\pi(x)$ число простых чисел, не превосходящих x. Существует оценка, что:
\[ \lim\limits_{x\to\infty}\frac{\pi(x)}{\frac{x}{\ln(x)}}=1\]
\begin{center}
    \begin{tikzpicture}[]
        \begin{axis}[
            title = Функция простых чисел и её приближение,
            xlabel = {$x$},
            ylabel = {$y$},
            xmin = 0, xmax = 1000,
            ymin = 0,
            ]
            \addplot [
                black,
                domain = 2:1000,
                samples = 200,
            ] {x/ln(x)};
            \addplot[
                black,
                scatter,
                only marks,
            ] table {prime_func_table.dat};
        \end{axis}
    \end{tikzpicture}
\end{center}
Приведённые выше формулы не помогут нам в вычислительных целях.
Чтобы полнее исследовать это явление, перепишем формулу для факториала в виде $\lim\limits_{n\to\infty}f(n)=1$.
Эта формула говорит нам только о существовании фукнции $N(\varepsilon)$ со следующим свойством:
\begin{center}
\[
\forall\varepsilon>0\forall n>N(\varepsilon)\Rightarrow |f(n)-1|<\varepsilon
\]
\end{center}
Знание фукнции $N(\varepsilon)$ на самом деле даёт некоторую численную информацию. Однако, используя обозначение $f(n)\to1$, мы
отбрасываем информацию о существовании конкретного вида такой функции на утверждения, что такая фукнция существует.
\newpage
\section{Большая O}
Если $S$-какое-либо множество, а $f$ и $\varphi$-действительные или комплексные функции, определённые на $S$, то формула
\begin{center}
\[f(s)=O(\varphi(s)),   s\in\mathcal{S}\]
\end{center}
означает, что 
\begin{center}
    \[\exists A = const>0: |f(s)|\leq A|\varphi(s)|,   s\in\mathcal{S}\]
\end{center}
если, в частности, $\varphi(s)\neq{0}, \forall{s\in\mathcal{S}}$, то из определения большого $O$ очевидно, что
\begin{center}
\[\frac{f(s)}{\varphi(s)}\ \textrm{ ограничено на } \mathcal{S}\]
\end{center}
Несколько примеров:
1. $x^{2} = O(x)$ $(|x|<2)$
\begin{center}
    \begin{tikzpicture}[]
        \begin{axis}[
            title = $x^2$,
            xlabel = {$x$},
            ylabel = {$y$},
            ]
            \addplot [
                black,
                domain = -2:2,
            ] {x^2};
            \addplot [
                red,
                domain = -2:2,
            ] {3*abs(x)};
        \end{axis}
    \end{tikzpicture}
\end{center}
2. $\sin(x) = O(1)$ $(-\infty<x<\infty)$
\begin{center}
    \begin{tikzpicture}[]
        \begin{axis}[
            title = $\sin(x)$,
            xlabel = {$x$},
            ylabel = {$y$},
            ]
            \addplot [
                black,
                domain = -4:4,
                samples = 400,
            ] {abs(sin(x*180/pi))};
            \addplot [
                red,
                domain = -4:4,
            ] {1};
        \end{axis}
    \end{tikzpicture}
\end{center}
3. $\sin(x) = O(x)$ $(-\infty<x<\infty)$
\begin{center}
    \begin{tikzpicture}[]
        \begin{axis}[
            title = $\sin(x)$,
            xlabel = {$x$},
            ylabel = {$y$},
            ]
            \addplot [
                black,
                domain = -4:4,
                samples = 400,
            ] {abs(sin(x*180/pi))};
            \addplot [
                red,
                domain = -4:4,
            ] {abs(x)};
        \end{axis}
    \end{tikzpicture}
\end{center}
Иногда свойство выполняется на некотором интервале, тогда, чтобы избавиться от этих мелких неприятностей, пользуются видоизменением обозначения О.
Объясним это значени для случая, когда нас интересуют большие положительные значения $x$ $(x\to\infty)$. Именно, мы будем писать $f(x)=O(\varphi(x))$   $(x\to\infty)$
Если $\exists{a}:f(x)=O(\varphi(x))$    $(a<x<\infty)$
Другими словами:
\begin{center}
\[\exists A>0 \exists a>0: |f(x)| \leq A|\varphi(x)| \textrm{при $a<x<\infty$}\]
\end{center}
Примеры:
1. $x^{2}=O(x)$     $(x\to0)$;
\begin{center}
    \begin{tikzpicture}[]
        \begin{axis}[
            title = $x^{2}$,
            xlabel = {$x$},
            ylabel = {$y$},
            ]
            \addplot [
                black,
                domain = -1:1,
                samples = 400,
            ] {x^2};
            \addplot [
                red,
                domain = -1:1,
            ] {abs(x)};
        \end{axis}
    \end{tikzpicture}
\end{center}
2. $e^{-x}=O(1)$     $(x\to\infty)$;
\begin{center}
    \begin{tikzpicture}[]
        \begin{axis}[
            title = $e^{-x}$,
            xlabel = {$x$},
            ylabel = {$y$},
            ]
            \addplot [
                black,
                domain = 0:20,
                samples = 200,
            ] {e^(-x)};
            \addplot [
                red,
                domain = 0:20,
            ] {0.5};
        \end{axis}
    \end{tikzpicture}
\end{center}   
3. ${\ln(x)}^{6}=O(\sqrt[2]{x})$     $(x\to\infty)$;
\begin{center}
    \begin{tikzpicture}[]
        \begin{axis}[
            title = $(\ln(x))^{6}$,
            xlabel = {$x$},
            ylabel = {$y$},
            ]
            \addplot [
                black,
                domain = 0.001:0.009,
                samples = 200,
            ] {ln(x)^6};
            \addplot [
                red,
                domain = 0.001:0.009,
                samples = 200,
            ] {1000000*sqrt(x)};
        \end{axis}
    \end{tikzpicture}
\end{center}   
Рассмотрим несколько примеров применения символа О:
\[O(x)+O(x^{2})=O(x)\quad(x\to0)\]
\[O(x)+O(x^{3})=O(x^{3})\quad(x\to\infty)\]
\[e^{O(1)}=O(1)\quad(-\infty<x<\infty)\]
\[e^{O(x)}=e^{O(x^{2})}\quad(x\to\infty)\]
\[\frac{1}{x}O(1)=O(1)+O(\frac{1}{x^{2}})\quad(0<x<\infty)\]
Последняя запись показывает, что если для фукнции $f(x)$ справедливо $f(x)=O(1)(0<x<\infty)$, то фукнцию $\frac{f(x)}{x}$ можно разбить на сумму двух слагаемых $g(x)$ и $h(x)$ таких,
что $g(x)=O(1)$,$h(x)=O(\frac{1}{x^{2}})$. \\ 
Доказать это можно просто: \\
положим $g(x)=0\quad\textrm{при }0<x\le1$, $g(x)=x^{-1}f(x)\quad\textrm{при }x>1$ \\
положим $h(x)=x^{-1}f(x)\quad\textrm{при }0<x\le1$, $h(x)=0\quad\textrm{при }x>1$ \\
Можно дать следующее толкование формул: любое выражение, содержащее символ $O$, следует рассматривать как класс фукнций. Например, на отрезке $0<x<\infty$ сумма $O(1)+O(x^{-2})$
означает класс всех фукнций вида $f(x)+g(x)$, где $f(x)=O(1) (0<x<\infty)$, $g(x)=O(x^{-2}) (0<x<\infty)$. Иногда в левой части может стоять отдельная фукнция. это означает, что она,
стоящяя в левой части, входит в класс, стоящий в правой части. \\
Знак равенства не совсем подходит для такого рода отношение, так как, например, соотношение \\
 $O(x)=O(x^{2})(x\to\infty)$ \\
справедливо, а соотношение \\ 
$O(x^{2})=O(x)(x\to\infty)$ не справедливо. \\
Пусть $\varphi$ и $\psi$-функции такие, что $\varphi(x)=O(\psi(x))(x\to\infty)$ выполняется и обратное неверно. \\
Если третья функция $f$ удовлетворяет условию $f(x)=O(\varphi(x))(x\to\infty)$, то она удовлетворяет условия $f(x)=O(\psi(x))(x\to\infty)$ \\
Если справедливо отношение $\varphi(x)=O(\psi(x))(x\to\infty)$, то мы назовём его уточнением соотношения $f(x)=O(\varphi(x))(x\to\infty)$ \\
Соотношение 
\\ $\varphi(x)=O(\psi(x))(x\to\infty)$ назовём наилучшим возможным, если оно не может быть уточнено, т.е. если \\ 
$\exists A\geq a>0: a|\varphi(x)|\le|f(x)|\le A|\varphi(x)| \forall x$ достаточно больших \\
Например, соотношение \\ 
$2x+x\sin(x)=O(x)\quad(x\to\infty)$ является наилучшим возможным, так как \\ 
$x\le2x+x\sin(x)\le3x$ \\
Также наилучшим соотноешнием является \\
$\ln(e^{2x\cos(x)}+e^{x})=O(x)\quad(x\to\infty)$, \\
т.к. $\ln(e{2x\cos(x)}+e^{x})\geq\ln(e^{x})=x$ и $\ln(e{2x\cos(x)}+e^{x})\leq\ln(e{2x}+e^{x})\leq\ln(2e^{2x})=2x+\ln2$ \\
Пусть $m\in\mathbb{Z}$, то можно привести оценку $e^{-x}=O(x^{-m})(x\to\infty)$
\begin{center}
    \begin{tikzpicture}[]
        \begin{axis}[
            title = \textrm{При m = 1,2,3},
            xlabel = {$x$},
            ylabel = {$y$},
            ]
            \addplot [
                black,
                domain = 10:20,
                samples = 50,
            ] {exp(-x)};
            \addplot [
                red,
                domain = 10:20,
                samples = 50,
            ] {x^(-1)};
            \addplot [
                green,
                domain = 10:20,
                samples = 50,
            ] {x^(-2)};
            \addplot [
                blue,
                domain = 10:20,
                samples = 50,
            ] {x^(-3)};
        \end{axis}
    \end{tikzpicture}
\end{center}
Однако ни одна из оценок не будет наилучшей возможной, т.к. всегда возможно улучшение \\ $e^{-x}=O(x^{-m-1})(x\to\infty)$ \\
Разберёмся с вопросом равномерности: \\
Пусть $S$-множество значений $x$, $k$-положительное число, $f(x)$ и $g(x)$-произвольные.Тогда \\
${(f(x)+g(x))}^{k}=O({(f(x))}^{k})+O({(g(x))}^{k})$ \\
В самом деле, \\
$|f+g|^{k}\le{(|f|+|g|)}^{k}\leq{(2\max(|f|,|g|))}^{k}\leq2^{k}\max(|f|^{k},|g|^{k})\leq2^{k}(|f|^{k}+|g|^{k})$ \\
Итого получаем, что $\exists A>0B>0: {|f(x)+g(x)|}^{k}\le A|f(x)|^{k}+B|g(x)|^{k}$, причём мы не знаем существуют ли $A$ и $B$, не зависящие от $k$. \\
С другой стороны, в соотношении \\
${(\frac{k}{x^{2}+k^{2}})}^{k}=O(x^{-k})\quad(1<x<\infty)$ \\
Причём постоянная $A$ может быть выбрана не зависящей от $k(0<k<\infty)$, т.к. \\
${(\frac{k}{x^{2}+k^{2}})}^{k}\le\frac{1}{{(2x)}^{k}}$ \\
$\forall k>0 \Rightarrow   2^{-k}<1$, следовательно, можно выбрать число $A$, не зависящее от k, так, чтобы \\
${(\frac{k}{x^{2}+k^{2}})}^{k}\le\frac{A}{x^{k}}\quad(1<x<\infty, k>0)$ \\
\begin{center}
    \begin{tikzpicture}[]
        \begin{axis}[
            title = \textrm{k=1, A=1},
            xlabel = {$x$},
            ylabel = {$y$},
            ]
            \addplot [
                black,
                domain = 1:20,
                samples = 50,
            ] {1/(x^2+1)};
            \addplot [
                red,
                domain = 1:20,
                samples = 50,
            ] {1/x};
        \end{axis}
    \end{tikzpicture}
\end{center}
\begin{center}
    \begin{tikzpicture}[]
        \begin{axis}[
            title = \textrm{k=sqrt(x), A=1},
            xlabel = {$x$},
            ylabel = {$y$},
            ]
            \addplot [
                black,
                domain = 1:20,
                samples = 50,
            ] {(sqrt(x)/(x^2+x))^sqrt(x)};
            \addplot [
                red,
                domain = 1:20,
                samples = 50,
            ] {(1/x)^sqrt(x)};
        \end{axis}
    \end{tikzpicture}
\end{center}
Этот факт можно выразить, сказав, что оценка равномерна по $k$. \\
Равномерность оценки важна в ситуации, когда мы хотим получить $O$-оценку для какой-нибудь фукнции, например \\
Пусть $f(x)$-фукнция, для который мы хотим получить оценку, и мы имеем для $f(x)$ некоторое выражение, которое мы разбиваем на два слагаемых, т.е. \\
$f(x)=O(x^{2}t)+O(x^{4}t^{-2})\quad(x>1,t>1)$, где $t$-параметр, от которого зависит разбиение \\
Далее мы хотим выбрать $t$ таким образом, чтобы правая часть стала наименьше возможной. Поскольку оценка равномерна, можно считать $t$ равным некоторой фукнции от $x$. \\
Это приводит к задаче: найти минимум $x^{2}t+x^{4}t^{-2}$ при данном $x$ \\
Решив эту задачу, получаем, что минимум достигается при $t={(2x^{2})}^{\frac{1}{3}}$, причём при таком $t$ оба слагаемые имеют один и тот же порядок.\\
Итого получаем, что $f(x)=O(x^{\frac{8}{3}})(x>1)$\\
В $O$-оценках, содержащих условия вида $x\to\infty$, имеются две постоянные $A$ и $a$. Мы будем говорить, что такая оценка равномерна по $k$ лишь в том случае, когда
обе постоянные $A$ и $a$ могут быть выбраны независимо от $k$. \\
Пример: \\
$\frac{k^{2}}{1+kx^{2}}=O(\frac{1}{x})\quad(x\to\infty)$ \\
Эта оценка не является равномерной по, т.к. иначе $\exists A>0 \exists a>0$ не зависящие от $k$ такие, что $\frac{k^{2}}{1+kx^{2}}<\frac{A}{x}\quad(x>a,k>0)$ \\
Однако тогда, положив $k=x^{2}$, мы получили бы, что $A(1+x^{4})>x^{5}$ при любом $x>a$, что невозможно.
\newpage
\section{Малая о}
Формула $f(x)=o(\varphi(x))\quad(x\to\infty)$ означает, что отношение $\frac{f(x)}{\varphi(x)}$ стремится к нулю при $x\to\infty$. \\
Это более сильная оценка, чем $O$-оценка. В асимптотических оценках символы $o$ имеют меньшее значение, чем $O$, поскольку они несут в себе меньше информации.
Если какая-либо величина стремится к нулю, мы хотим знать, с какой скоростью это происходит.
\newpage
\section{Асимптотическое равенство}
Мы будем говорить, что $f(x)$ и $g(x)$ асимптотически равны при $x\to\infty$, 
если отношение $\frac{f(x)}{g(x)}$ стремится к единице. Записывать этот факт мы будем формулой $f(x)\mathtt{\sim}g(x)$ \\
Это обозначение будет также использоваться и при любом другом способе стремления переменной к пределу. \\
Примеры: \\
1. $x+1\mathtt{\sim}x\quad(x\to\infty)$ \\
2. $sh(x)\mathtt{\sim}\frac{1}{2}e^{x}\quad(x\to\infty)$ \\
3. $n!\mathtt{\sim}{(\frac{n}{e})}^{-n}\sqrt[]{2n\pi}\quad(n\to\infty)$ \\
Говоря об "асимптотическим поведением" данной фукнции $f(x)$, можно иметь в виду асимптотическую информацию любого рода. Однако обычно подразумевают "простую" фукнцию $g(x)$
асимптотически равную $f(x)$. Здесь "простая" означает, что способ точного вычисления её значений не становится исключительно сложен, когда $x$ очень велико. \\
Слова "асимптотическая формула для $f(x)$" обычно употребляются в том же узком смысле, т.е. в них речь идёт о формуле $f(x)\mathtt{\sim}g(x)$.
\newpage
\section{Асимптотические ряды}
Часто бывает так, что для фукнции $f(x)$ при $x\to\infty$ и  меется бесконечная последовательность $O$-оценок, причём каждая следующая оценка как бы усовершенствует предыдущую.
Особенно часто встречается последовательность такого вида: $\varphi_0, \varphi_1, \varphi_2, \dots$, удовлетворяющих условиям \\
$\varphi_1(x)=o(\varphi_0(x))(x\to\infty),\varphi_2(x)=o(\varphi_1(x))(x\to\infty),\dots$ \\
и последовательность постоянных $c_0,c_1,c_2,\dots$ таких, что для $f(x)$ имеет место последовательность $O$-оценок: \\
\[
\left\{
\begin{array}{l}
    f(x)=O(\varphi_0(x))\quad{(x\to\infty)} \cr
    f(x)=c_0\varphi_0(x)+O(\varphi_1(x))\quad(x\to\infty) \cr
    f(x)=c_0\varphi_0(x)+c_1\varphi_1(x)+O(\varphi_2(x))\quad(x\to\infty) \cr
    \cdots \cr
    f(x)=c_0\varphi_0(x)+c_1\varphi_1(x)+c_2\varphi_2(x)+\dots+c_{n-1}\varphi_{n-1}(x)+O(\varphi_n(x))\quad(x\to\infty) \cr
    \cdots
\end{array}
\right.
\]
Очевидно, что вторая формула усовершенствует первую, поскольку $c_0\varphi_0(x)+O(\varphi_1(x))=(c_0+o(1))\varphi_0(x)=O(\varphi_0(x))(x\to\infty)$, аналогично и для других \\
Чтобы записать все множество формула одной формулой, воспользуемся следующим обозначением: \\
$f(x)\approx c_0\varphi_0(x)+c_1\varphi_1(x)+c_2\varphi_2(x)+\dots\quad(x\to\infty)$ \\
Правую часть этого выражения мы назовём асимптотическим рядом для $f(x)$, или асимптотическим разложением фукнции $f(x)$. \\
Нетрудно убедиться, что при данных $\varphi_k$ и $f$ величины $c_k$ определяются единственным образом, если асимптотическое разложение $f$ по $\varphi_k$ существует. \\
Допустим, что имеется другой асимптотический ряд:
$f(x)\approx d_0\varphi_0(x)+d_1\varphi_1(x)+d_2\varphi_2(x)+\dots\quad(x\to\infty)$ \\
Обозначим за $k$-наименьшее число такое, что $c_k\neq d_k$, после вычитания получим: \\
$0=(c_k-d_k)\varphi_k(x)+O(\varphi_{k+1}(x))\Rightarrow\varphi_k(x)=O(\varphi_{k+1}(x))$
А это противоречит условию, что $\varphi_{k+1}(x)=o(\varphi_k(x))$ \\
Иногда вычисленные коэффициенты равны нулю, тогда условимся писать: \\
$f(x)\approx 0\cdot\varphi_0(x)+0\cdot\varphi_1(x)+0\cdot\varphi_2(x)+\dots\quad(x\to\infty)$ \\
Это значит, что $f(x)=O(\varphi_n(x))(x\to\infty)\forall n\quad\textrm{(но не обязательно равномерно по n)}$ \\
Например, поскольку $e^{-x}=O(x^{-n})(x\to\infty)\forall n$, то мы можем написать \\
$e^{-x}\approx0\cdot{1}+0\cdot{x^{-1}}+0\cdot{x^{-2}}+\dots$ \\
Асимптотический ряд не обязательно сходится. Причина этого заключается в том, что сходимость является некоторым свойством ряда при фиксированном $x_0$,
в то время как $O$-оценки относятся не к фиксированному $x$, а к $x\to\infty$. Сходимость асимптотического ряда, $\forall{x>0}$, означает, что $\forall{x}$
ряд обладает некоторым свойством при $n\to\infty$. С другой стороны, утверждение, что ряд является асимптотическим разложением функции $f(x)$, означает, что
этот ряд обладает тем же свойством при фиксированном $n$ и при $x\to\infty$. \\
Более того, даже если асимптотический ряд сходится, его сумма не обязана быть равной $f(x)$ (пример с $e^{-x}$). Можно даже подобрать такие фукнции $f(x),\varphi_i(x)$
таким образом, чтобы асимптотический ряд сходился $\forall{x}$ и в то же время не являлся бы асимптотическим рядом своей суммы. \\
Рассмотрим пример расходящегося асимптотического ряда: \\
\begin{center}
$f(x)=\int_{1}^{x}\frac{e^{t}}{t}\,dt$ \\
$f(x)=\left.\frac{e^{t}}{t}\right\vert_{1}^{x}+\int_{1}^{x}\frac{e^{t}}{t^{2}}\,dt$ \\
$\int_{1}^{\frac{x}{2}}t^{-2}e^{t}\,dt<\int_{1}^{\frac{x}{2}}e^{t}\,dt<e^{\frac{x}{2}}$ \\
$\int_{\frac{x}{2}}^{x}t^{-2}e^{t}\,dt<\int_{\frac{x}{2}}^{x}{\left(\frac{x}{2}\right)}^{-2}e^{t}\,dt<4x^{-2}e^{x}$ \\
Видно, что $e=O(x^{-2}e^{x})$, $e^{\frac{x}{2}}=O(x^{-2}e^{x})$, $4x^{-2}e^{x}=O(x^{-2}e^{x})$
    \begin{tikzpicture}[]
        \begin{axis}[
            title = \textrm{Черным $\frac{e^{x}}{x^{2}}$, красным $e$ и $e^{\frac{x}{2}}$},
            xlabel = {$x$},
            ylabel = {$y$},
            ]
            \addplot [
                black,
                domain = 1:30,
                samples = 50,
            ] {exp(x)/x^2};
            \addplot [
                red,
                domain = 1:30,
                samples = 50,
            ] {e};
            \addplot [
                red,
                domain = 1:30,
                samples = 50,
            ] {exp(x/2)};
        \end{axis}
    \end{tikzpicture}
А потому можно сказать, что $f(x)=\frac{e^{x}}{x}+O\left(\frac{e^{x}}{x^{2}}\right)\quad(x\to\infty)$ \\
\end{center}
Дальнейшее усовершенствование оценки можно получить, повторяя ту же операцию интегрирования по частям: \\
\begin{center}
$f(x)=\left.\frac{e^{t}}{t}\right\vert_{1}^{x}+\left.\frac{e^{t}}{t^{2}}\right\vert_{1}^{x}+\int_{1}^{x}\frac{2e^{t}}{t^{3}}\,dt$ \\
$f(x)=\left.\frac{e^{t}}{t}\right\vert_{1}^{x}+\left.\frac{e^{t}}{t^{2}}\right\vert_{1}^{x}+\left.\frac{2e^{t}}{t^{3}}\right\vert_{1}^{x} +\int_{1}^{x}\frac{3!e^{t}}{t^{4}}\,dt$ \\
Итого можно получить:
$f(x)=\left.e^{t}\left(\frac{1}{t}+\frac{1!}{t^{2}}+\frac{2!}{t^{3}}+\dots+\frac{(n-1)!}{t^{n}}\right)\right\vert_{1}^{x}+\int_{1}^{x}\frac{n!e^{t}}{t^{n+1}}\,dt$
\end{center}
Последний интеграл равен $O(x^{-n-1}e^{x})$ при $x\to\infty$ и при фиксированном $n$. Это можно доказать, разбив его на две части, а именно на $\left(1,\frac{x}{2}\right)$ 
и $\left(\frac{x}{2},x\right)$. \\
Итого при каждом $n$ имеем:\\
$\frac{f(x)}{e^{x}}=\frac{1}{x}+\frac{1}{x^{2}}+\frac{2!}{x^{3}}+\dots+\frac{(n-1)!}{x^{n}}+O\left(\frac{1}{x^{n+1}}\right)$ \\
откуда следует: \\
$\frac{f(x)}{e^{x}}\approx\frac{1}{x}+\frac{1}{x^{2}}+\frac{2!}{x^{3}}+\frac{3!}{x^{4}}+\dots$ \\
Ряд в правой части не сходится ни при одном значении $x$.  
Простым и тривиальным классом асимптотических рядов является класс сходящихся степенных рядов. \\
Пусть $f(z)$-сумма сходящегося степенного ряда $f(z)=a_0+a_1z+a_2z^{2}+\dots$ \\
причём $|z|\le\rho$, где $\rho>0$ и меньше радиуса сходимости, тогда \\
$f(x)\approx a_0+a_1z+a_2z^{2}+\dots\quad(|z|\to 0)$
Доказать это очень просто. \\
Из сходимости ряда при $z=\rho\Rightarrow|a_n|\rho^{n}\le A\forall n$ \\

\begin{multline}
        \forall n \forall z: |z|\le\frac{\rho}{2}\Rightarrow{\left\vert\sum_{k=n+1}^{\infty}a_k z^{k}\right\vert}\le 
        \sum_{k=n+1}^{\infty}|a_k||z|^{k}=
        |z|^{n+1}\sum_{k=n+1}^{\infty}|a_k||z|^{k-n-1}\le \\
        \le |z|^{n+1}\sum_{k=n+1}^{\infty}|a_k|\left(\frac{\rho}{2}\right)^{k-n-1}=
        |z|^{n+1}\left(\frac{\rho}{2}\right)^{n+1}\left(\frac{2}{\rho}\right)^{n+1}\sum_{k=n+1}^{\infty}|a_k|\left(\frac{\rho}{2}\right)^{k-n-1} = \\
        =\left(\frac{|z|}{\rho}\right)^{n+1}\sum_{k=n+1}^{\infty}|a_k|\rho^{k}\frac{1}{2^{k-n-1}} =
        A\left(\frac{|z|}{\rho}\right)^{n+1}\sum_{k=n+1}^{\infty}\frac{1}{2^{k-n-1}} \le \\
        \le A\left(\frac{|z|}{\rho}\right)^{n+1}\sum_{k=0}^{\infty}\frac{1}{2^{k}} = 2A\left(\frac{|z|}{\rho}\right)
\end{multline}
откуда: \\
$f(z)=a_0+a_1z+\dots+a_nz^{n}+O(z^{n+1})\quad(|z|<\frac{\rho}{2})$ \newpage
\section{Элементарные действия с асимптотическими рядами}
Для простоты ограничимся асимптотическими рядами вида \\
$a_0+a_1x+a_2x^{2}+\dots\quad(x\to{0})$ \\
хотя аналогичные выводы можно сделать и для рядов других видов. \\
Рассматриваемый ряд является степенным рядом и независимо от его сходимости мы будем называть его формальным степенным рядом. \\
Если для таких рядов определить сложение и умножение, то множество таких рядов станет коммутативным кольцом, единицей которого будет $I=1+0\cdot{x}+0\cdot{x^{2}}+\dots$ \\
$A=a_0+a_1x+a_2x^{2}+\dots$ и $B=b_0+b_1x+b_2x^{2}+\dots$ \\
Определим сумму и произведение равенствами: \\
$A+B=(a_0+b_0)+(a_1+b_1)x+(a_2+b_2)x^{2}+\dots$ \\
$A\cdot{B}=a_0b_0+(a_0b_1+a_1b_0)x+(a_0b_2+a_1b_1+a_2b_0)x^{2}+\dots$ \\
Если $a_0\neq{0}$, то $\exists!{C}: A\cdot{C}=I$. Его коэффициенты $c_0,c_1,c_2,\dots$ определяются из уравнений: \\
$a_0c_0=1,\quad a_0c_1+a_1c_0=0, a_0c_2+a_1c_1+a_2c_0=0,\dots$
Предположим, что $b_0=0$, тогда можно определить формальный степенной рял, получающийся в результате подстановки ряда $B$ в ряд $A$, этот ряд обозначим $A(B)$.\\
Определим его следующим образом: \\
Пусть $c_{kn}$-коэффициент при $x^{k}$ в ряде $a_0I+a_1B+a_2B^{2}+\dots+a_nB^{n} = c_0 +c_1x+\dots+c_nx^{n}+c_{n+1}x^{n+1}+c_{n+2}x^{n+2}+\dots$ \\
Следующей операцией над формальными рядами является дифференцирование. Производную ряда $A=a_0+a_1x+a_2x^{2}+\dots$ определим формулой: \\
$A^{\prime}=a_1+2a_2x+3a_3x^{2}+\dots$ \\
Если $A$ и $B$-степенные ряды с отличным от нуля радиусом сходимости, то все эти формальные действия в точности соответсвтуют тем же действиям над суммами $A(x)$ и $B(x)$ этих рядов.\\
Например, если $A(B)=C$, то ряд $C$ имеет отличный от нуля радиус сходимости, и внутри круга этого радиуса $A\left\{B(x)\right\}=C(x)$. \\
Если говорить об асимптотических рядах вместо сходящихся степенных рядов, то мы имеем совершенно аналогичное положение, за исключением того, что вопрос о дифференцировании требует
особой осторожности. \\
Пусть $A(x)$ и $B(x)$-фукнции, определённые в окрестности $x=0$ и имеющие асимптотические разложения $A(x)\approx{A}(x\to{0})$, $B(x)\approx{B}(x\to{0})$, тогда:
$A(x)+B(x) = (a_0+b_0)+(a_1+b_1)x+(a_2+b_2)x^{2}+\cdots+O(x^{n})\approx(a_0+b_0)+(a_1+b_1)x+(a_2+b_2)x^{2}+\cdots = A+B\quad(x\to{0})$ \\
$A(x)B(x)=(a_0+\cdots+a_nx^{n})(b_0+\cdots+b_nx^{n})+O(x^{n+1})=c_0+c_1x+c_2x^{2}+\cdots+O(x^{n+1})\approx c_0+c_1x+c_2x^{2}+\cdots=AB\quad(x\to{0})$ \\
$\left\{A(x)\right\}^{-1}\approx{A^{-1}}\quad(x\to{0})$ \\
$A\left\{B(x)\right\}\approx{A(B)}\quad(x\to{0})$ \\
Предположим, что $f(x)\approx{a_0+a_1x+a_2x^{2}+\dots}\quad(x\to{0})$ и что $\exists{\int_{0}^{x}{f(t)\,dt}}\forall{x \textrm{достаточно малых}}$, тогда законно почленное интегрирование: \\
$\int_{0}^{x}{f(t)\,dt}\approx{a_0x+\frac{1}{2}a_1x^{2}+\frac{1}{3}a_2x^{3}+\dots}\quad(x\to{0})$ \\
Это легко доказать. $\forall{n}\exists{A>0}\exists{a>0}:|f(t)-a_0-a_1t-\cdots-a_{n-1}t^{n-1}|<A|t|^{n}\quad(|t|<a)$ откуда при $|x|<a$ получаем: \\
$\left\vert{\int_{0}^{x}{f(t)\,dt}-a_0x-\frac{1}{2}a_1x^{2}-\cdots-\frac{1}{n}a_{n-1}x^{n}}\right\vert<\frac{A}{n+1}|x|^{n+1}$ и соотношение доказано. \\
При дифференцировании даже, если $A(x)$ имеет асимптотическое разложение, то производная $A^{\prime}(x)$ не обязательно существует, а если она и существует, то может не иметь
асимптотического разложения. Однако почленное дифференцирование асимптотических рядов все же является законным, если удается показать, что производная тоже имеет асимптотическое
разложение. Действительно, предположим, что \\
$f(x)\approx{a_0+a_1x+a_2x^{2}+\dots}\quad(x\to{0})$ \\
$f^{\prime}(x)\approx{b_0+b_1x+b_2x^{2}+\dots}\quad(x\to{0})$ \\
$g_n(x)=f(x)-(b_0x+\frac{1}{2}b_1x^{2}+\dots+\frac{1}{n}b_{n-1}x^{n}) (n>0)$ \\
$g_n^{\prime}(x)=O(x^{n})\quad(x\to{0})$
Из теоремы о среднем следует, что $g_n(x)-g_n(0)=O(x^{n+1})(x\to{0})$ \\
Поскольку $n$ произвольно, то: \\
$f(x)\approx{f(0)+b_0x+\frac{1}{2}b_1x^{2}+\frac{1}{3}b_2x^{3}+\dots}\quad(x\to{0})$ \\
Теперь формула $b_k=(k+1)a_{k+1}$ следует из единственности коэффициентов асимптотического ряда.
\newpage
\section{Неявные фукнции}
Пусть фукнциональная зависимость $x$ от $t$ задана уравнением $f(x,t)=0$, причём если уравнение имеет больше одного корня, то для каждого значения $t$ указано, какой из корней
должент быть выбран. Этот корень мы будем обозначать $x=\varphi(t)$. Задача состоит в определении асимптотического поведения функции $\varphi(t)(t\to\infty)$. \\
В общем случае задача довольно неопределённа, так на самом деле мы хотим выразить асимптотическое поведение данной фукнции $\varphi(t)$ в терминах элементарных функций или по
крайней мере в терминах явных функций. При этом существенно, какие фукнции считать элементарными. Во многих встречающихся в практике случаях можно выразить асимптотическое поведение
неявной фукнции в терминах элементарных функций. Приведём любопытный пример, когда такого выражения нет. Если $x$ задано уравнением: \\
$x(\ln{x})^{t}-t^{2t}=0\quad(x>1)$ \\
то легко убедиться, что $x=e^{t\varphi(t)}$, где $\varphi(t)$-решение уравнения $\varphi e^{\varphi}=t$. При $t\to\infty$ мы имеем для $\varphi$ асимптотическое разложение, позволяющее
определить $\varphi(t)$ с ошибкой порядка $(\ln{t})^{-k}$, где $k$ произвольное, но фиксированное число. Это значит, что мы имеем асимптотическую формулу для $\ln{x}$, но не для $x$.
Иными словами, у нас нет такой элементарной фукнции $\psi(t):\frac{x}{\psi(t)}\to{1}$ при $t\to\infty$, иначе бы существовала оценка с ошибкой порядка $o(t^{-1})$.
\newpage
\section{Формула обращения Лагранжа}
Пусть фукнция $f(z)$ аналитична в некоторой окрестности $z=0$ комплексной плоскости. Предположив, что $f(0)\neq{0}$, рассмотрим уравнение: \\
$\omega=\frac{z}{f(z)}$, где $z$-неизвестное \\
$\exists{a>0}\exists{b>0}:$ при $|\omega|<a$ уравнение имеет единственное решение в области $|z|<b$ и это решение является аналитической фукнцией $\omega$: \\
$z=\sum_{k=1}^{\infty}c_k\omega^{k}\quad(|\omega|<a)$ \\
при этом коэффициенты можно найти по формулам:
$c_k=\frac{1}{k!}\left\{\left(\frac{d}{dz}\right)^{k-1}(f(z))^{k}\right\}_{z=0}$ \\
Обобщённая формула даёт значения $g(z)$, где $g$-любая фукнция $z$, аналитическая в окрестности точки $z=0$: \\
$g(z)=g(0)+\sum_{1}^{\infty}d_k\omega^{k}$ \\
$d_k=\frac{1}{k!}{\left(\frac{d}{dz}\right)}^{k-1} \left\{g^{\prime}(x)(f(z))^{k}\right\}_{z=0}$ \\
Формула $z=\sum_{k=1}^{\infty}c_k\omega^{k}\quad(|\omega|<a)$ обычно называемя формулой обращения Лагранжа, является частным случаем более обшей теоремы о неявной фукнции. \\
Если $f(z,\omega)$-аналитическая фукнция $z$ и $\omega$ в некоторой области $|z|<a_1$, $|\omega|<b_1$, причём $f(0,0)=0$ и $\frac{\partial{f}}{\partial{z}}\neq{0}$ при $z=\omega=0$, то
существуют положительные числа $a$ и $b$, такие, что для любого $\omega$ в круге $|\omega|<a$ уравнение $f(z,\omega)=0$ имеет единственное решение в круге $|z|<b$ и это решение может быть
разложено в степенной ряд: \\
$z=\sum_{k=1}^{\infty}c_k\omega^{k}$ \newpage
\section{Применения}
Рассмотрим положительные решения уравнения $xe^{x}=t^{-1}$ при $t\to\infty$. Так как $t^{-1}$ стремится к нулю, можно применить формулу Лагранжа к уравнению $ze^{z}=\omega$, при этом
$f(z)=e^{-z}$, т.е. $\omega=\frac{z}{e^{-z}}$. Тогда можно утверждать, что $\exists{a>0}\exists{b>0}:$ при $|w|<a\exists!$ решение $z:|z|<b$ и $z=\sum_{k=1}^{\infty}\frac{(-1)^{k-1}k^{k-1}}{k!}\omega^{k}$. \\
Ряд сходится при $|\omega|<e^{-1}$. Отсюда , что при $t>a^{-1}$ существует единственное решение в круге $|x|<b$. Но поскольку $xe^{x}$ возрастает от $0$ до $\infty$, когда $x$ возрастает
от $0$ до $\infty$, уравнение имеет положительное решение и оно не может превосходить $b$, если $t$ достаточно далеко. Значит, при достаточно больших $t$ это положительное решение разлагается
в ряд: \\
$x=\sum_{k=1}^{\infty}\frac{(-1)^{k-1}k^{k-1}}{k!}t^{-k}$ \\
и этот степенной ряд служит также асимптотическим разложением.\\
Вторым примером будет положительное решение уравнения \\
$x^{t}=e^{-x}$ при $t\to\infty$ \\
Функция $x^{t}$ возрастает при $x>0$, а функция $e^{-x}$ убывает. Если заметить, что $x^{t}$ мало на отрезке $0\leq x\leq 1$, за исключением $x$, очень близких к 1, то из рассмотрения графиков $x^{t}$ и $e^{-x}$
становится ясно, что наше уравнение имеет ровно один положительный корень, который не превосходит 1 и стремится к 1 при $t\to\infty$. \\
ВСТАВИТЬ ГРАФИК \\
Положим теперь $x=1+z$, $t^{-1}=\omega$ и получим уравнение: \\
$f(z)=-\frac{z(1+z)}{\ln(1+z)}$ \\
Фукнция $f(z)$ аналитична в $z=0: f(z)=-1+c_1z+\dots$, следовательно: \\
$x=1-\frac{1}{t}-c_1\frac{1}{t^{2}}+\dots$\\
удовлетворяет уравнению при достаточно больших $t$. Существование единственного положительного решения, стремящегося к 1 при $t\to\infty$, обеспечивает разложимость этого решения в
степенной ряд при достаточно больших $t$. \\
Рассмотрим уравнение: \\
$\cos{x}=x\sin{x}$\\
Из графиков функций $x$ и $\coth{x}$ видно, что это уравнение имеет ровно по одному корню в каждом из интервалов $\pi{n}<x<\pi{(n+1)}$. \\
ВСТАВИТЬ ГРАФИК\\
Обозначая эти корни через $x_n$, поставим вопрос об асимптотическом поведении $x_n$ при $n\to\infty$. Так как $\coth{(x_n-\pi{n})}=x_n$ при $x_n\to\infty$, имеем $x_n-\pi{n}\to{0}$.
Полагая $x=\pi{n}+z$,$(\pi{n})^{-1}=\omega$, находим, что $\cos{z}=(\omega{-1}+z)\sin{z}$ и, следовательно: \\
$f(x)=\frac{z(\cos{z}-z\sin{z})}{\sin{z}}$, где $f(z)$ аналитична в точке $z=0$ и $f(0)=1$. \\
Поэтому $z$ разлагается в степенной ряд по степеням $\omega$, и мы получаем $z=\omega+c_2\omega^{2}+c_3\omega^{3}+\dots$. Следовательно, при достаточно большом $n$: \\
$x_n=\pi{n}+\frac{1}{\pi{n}}+\frac{c_2}{(\pi{n})^{2}}+\dots$ \\
Заметим, что $c_2=c_4=c_6=\dots=0$, т.к. $f(z)$-четная фукнция. \\
Рассмотрим уравнение $xe^{x}=t$, которое при положительном $t$ имеет единственное положительное решение $x$, поскольку фукнция $xe^{x}$ возрастает от $0$ до $\infty$. Это решение мы будем
обозначать просто $x$ и будем интересоваться его поведением при $t\to\infty$. \\
Преобразовать это уравнение трудно, поэтому будет использовать метод итераций. Запишем уравнение в виде: $x=\ln{t}-\ln{x}$ \\
Имея какое-либо приближенное выражение для $x$, мы можем подставить его в правую часть уравнения $x=\ln{t}-\ln{x}$ и получить новое приближение, более точное, чем прежнее. Заметим, что
погрешность $\varDelta$ в значении $x$ даёт нам погрешность примерно $\frac{\varDelta}{x}$ в значении $\ln{x}$. \\
Поскольку $t\to\infty$, то можно считать, что $t>e$ и, следовательно, $x>1$. Действительно, при $0<x\le{1}$ мы имели бы $\ln{t}-\ln{x}\le{\ln{t}}>\ln{e}=1$, в то время как левая часть $x=\ln{t}-\ln{x}$
по предположению не превосходит единицы. \\
Из неравенства $x>1$ следует, что $x=\ln{t}-\ln{x}<\ln{t}$, и мы начинаем с неравенства $1<x<\ln{t}$. \\
Отсюда в силу уравнения $x=\ln{t}-\ln{x}$ следует $\ln{x}=O(\ln\ln{t})$. \\
Следовательно $x=\ln{t}+O(\ln\ln{t})\quad(t\to\infty)$ \\
Логарифмируя находим, что \\
$\ln{x}=\ln\ln{t}+\ln\left(1+O(\frac{\ln\ln{t}}{\ln{t}})\right)=\ln\ln{t}+O\left(\frac{\ln\ln{t}}{\ln{t}}\right)$ \\
Подставив в уравнение получаем: \\
$x=\ln{t}-\ln\ln{t}+O\left(\frac{\ln\ln{t}}{\ln{t}}\right)$ \\
Снова логарифмируя полученное уравнение, получаем третье приближение: \\
\begin{multline}
x=\ln{t}-\ln\left(\ln{t}-\ln\ln{t}+O\left(\frac{\ln\ln{t}}{\ln{t}}\right)\right) = \\ 
= \ln{t}-\ln\ln{t}+\frac{\ln\ln{t}}{\ln{t}}+\frac{1}{2}\left(\frac{\ln\ln{t}}{\ln{t}}\right)^{2}+O\left(\frac{\ln\ln{t}}{(\ln{t})^{2}}\right) \\
\end{multline}
Введём сокращённые обозначения $\ln{t}=L_1,\quad\ln\ln{t}=L_2$, тогда получаем \\
$\ln{x}=L_2+\ln\left(1-\frac{L_2}{L_1}+\frac{L_2}{L_1^{2}}+\frac{1}{2}\frac{L_2^{2}}{L_1^{3}}+O\left(\frac{L_2}{L_1^{3}}\right)\right)$
поскольку член $O\left(\frac{L_2}{L_1^{3}}\right)$ поглащает все члены вида $\frac{L_2^{p}}{L_1^{q}}$ при $q>3$, имеем \\
\begin{multline}
    x=L_1-L_2-\left(-\frac{L_2}{L_1}+\frac{L_2}{L_1^{2}}+\frac{1}{2}\frac{L_2^{2}}{L_1^{3}}+O\left(\frac{L_2}{L_1^{3}}\right)\right)
    +\frac{1}{2}\left(-\frac{L_2}{L_1}+\frac{L_2}{L_1^{2}}\right)^{2}-\frac{1}{3}\frac{L_2^{3}}{L_1^{3}} = \\
    = L_1-L_2+\frac{L_2}{L_1}+\left(\frac{1}{2}L_2^{2}-L_2\right)\frac{1}{L_1^{2}}+\left(-\frac{1}{3}L_2^{3}-\frac{3}{2}L_2^{2}+O(L_2)\right)\frac{1}{L_1^{3}}
\end{multline}
На следующем шагу получим \\
\begin{multline}
x=L_1-L_2+\frac{L_2}{L_1}+\left(\frac{1}{2}L_2^{2}-L_2\right)\frac{1}{L_1^{2}}+\left(-\frac{1}{3}L_2^{3}-\frac{3}{2}L_2^{2}+L_2\right)\frac{1}{L_1^{3}}+\\
+\left(\frac{1}{4}L_2^{4}-\frac{11}{6}L_2^{3}+3L^{2}+O(L_2)\right)\frac{1}{L_1^{4}}
\end{multline}
При взгяде на эти формулы создаётся впечатление, что существует асимптотический ряд \\
$x\approx L_1-L_2+L_2P_0(L_2)\frac{1}{L_1}+L_2P_1(L_2)\frac{1}{L_1^{2}}+L_2P_2(L_2)\frac{1}{L_1^{3}}+\dots$, где $P_k(L_2)$-многочлен степени k \\
Это можно доказать, тщательно исследуя операции, которые приводят к полученному выражению и к последующим приближенным формулам такого вида. \\
Однако мы пойдём иным путём, а именно докажем, что если $t$ достаточно велико, то $x$ представляет собой сумму сходящегося ряда такого вида.
Для этого нам понадобится теорема Руше: \\
Пусть $D$-ограниченная область комплексной плоскости, её граница $C$-замкнутая жорданова кривая. Пусть, далее, фукнции $f(z)$ и $g(z)$ аналитичны в $D$ и на $C$,
причём $|f(z)|<|g(z)|$ на $C$. Тогда $f(z)+g(z)$ имеет в $D$ то же число нулей, что и $g(z)$, считая все нули с их кратностью. \\
Наш метод исследования уравнения $x=\ln{t}-\ln{x}$ построен по образцу обычного доказательства теоремы Лагранжа. Для сокращения записи введём обозначения \\
\begin{center}
$x=\ln{t}-\ln\ln{t}+v$
$\frac{1}{\ln{t}}=\sigma,\quad\frac{\ln\ln{t}}{\ln{t}}=\tau$
\end{center}
В этих обозначениях получается \\
$e^{-v}-1-\sigma{v}+\tau=0$ \\
На время мы забудем о связи между $\tau$ и $\sigma$, и будем рассматривать их как малые независимые комплексные параметры. Мы покажем, что $\exists{a>0}\exists{b>0}:|\sigma|<a,|\tau|<a$
уравнение имеет единственное решение в области $|v|<b$ и что это решение является аналитической функцией $\sigma$ и $\tau$ в области $|\sigma|<a, |\tau|<a$. \\
Обозначим через $\sigma=\inf_{z\in\{|z|=\pi\}}|e^{-z}-1|$. Ясно, что $\sigma>0$, а $e^{-z}-1$ имеет ровно одни нуль внутри этой окружности, именно $z=0$.
Затем выберем положительное число $a=\frac{\sigma}{2}(\pi+1)$. Тогда \\
$|\sigma{z}-\tau|<\frac{\sigma}{2}\quad(|\sigma|<a,|\tau|<a,|z|=\pi)$ \\
Отсюда $|e^{-z}-1|>|\sigma{z}-\tau|$ на окружности $|z|=\pi$, и по теореме Руше уравнение $e^{-z}-1-\sigma{z}+\tau=0$ имеет ровно один корень в круге $|z|<\pi$.
Обозначая этот корень через $v$, имеем по теореме Коши \\
$v=\frac{1}{2\pi{i}}\int_{|z|=\pi}\frac{-e^{-z}-\sigma}{e^{-z}-1-\sigma{z}+\tau}z\,dz$ \\
Для всех $z$ на пути интегрирования $|\sigma{z}|+|\tau|<\frac{1}{2}|e^{-z}-1|$, так что можно написать разложение в ряд \\
$\frac{1}{e^{-z}-1-\sigma{z}+\tau}=\sum_{k=0}^{\infty}\sum_{m=0}^{\infty}(-1)^{m}\frac{(m+k)!}{m!k!}\frac{z^{k}\sigma^{k}\tau^{k}}{(e^{-z}-1)^{k+m+1}}$ \\
сходящийся абсолютно и равномерно, когда $|z|=\pi,|\sigma|<a,|\tau|<a$. Следовательно, можно подставить этот ряд в формулу $v=\frac{1}{2\pi{i}}\int_{|z|=\pi}\frac{-e^{-z}-\sigma}{e^{-z}-1-\sigma{z}+\tau}z\,dz$ 
и проинтегрировать почленно, что даст нам выражение для $v$ в виде абсолютно сходящегося двойного степенного ряда. Заметим, что члены, не содержащие $\tau$, отсуствуют.
Таким образом, мы доказзали, что при $|\tau|<a,|\tau|<a$ уравнение $e^{-v}-1-\sigma{v}+\tau=0$ имеет единственное решение $v$, удовлетворяющее условию $|v|<\pi$, и это решение имеет вид \\
$v=\tau\sum_{k=0}^{\infty}\sum_{m=0}^{\infty}c_{km}\sigma^{k}\tau^{m}$, где $c_{km}$-постоянные. \\
Для достаточно больших $t$ имеем $|\sigma|=|\frac{1}{\ln{t}}|<a, |\tau|=|\frac{\ln\ln{t}}{\ln{t}}|<a$, кроме того, решение, которое нам нужно, мало: из оценки $x=\ln{t}-\ln\ln{t}+O\left(\frac{\ln\ln{t}}{\ln{t}}\right)$
следует, что $v=O\left(\frac{\ln\ln{t}}{\ln{t}}\right)$. Это значит, что оно совпадает при больших $t$ с найденным решением. Итаг, окончательный результат \\
$x=\ln{t}-\ln\ln{t}+\sum_{k=0}^{\infty}\sum_{m=0}^{\infty}c_{km}(\ln\ln{t})^{m+1}(\ln{t})^{-k-m-1}$  и ряд абсолютно сходится для всех достаточно больших значений $t$.
\newpage
\section{Метод итераций}
Пусть мы хотим знать асимптотическое поведение некоторой фукнции $f(t)$ при $t\to\infty$. Обычно, прежде чем начинать что-либо доказывать, очень важно иметь какие-то разумные предположения об этом поведении.
И чем лучше мы угадаем аппроксимируем для $f(t)$, тем легче доказать, что это и в самом деле есть некоторая аппроксимация. \\
Пусть $\varphi_0(t),\varphi_1(t),\dots$-последовательность фукнций, и предположим, что асимптотическое поведение $\varphi_k(t)$ при каждом отдельном $k$ известно. \\
Пусть, далее, мы имеем основания полагать, что свойства $\varphi_0(t)$ в некотором смысле близки к свойствам $f(t)$. Предположим ещё, что имеется операция, преобразующая $\varphi_0$ в $\varphi_1$,
$\varphi_1$ в $\varphi_2$ и т.д., и мы имеем основания полагать, что эта операция превращает хорошее приближение в ещё лучшее. При этом мы надеемся на то, что $\varphi_k$ может при некотором $k$ оказаться
настолько близким к $f$ (в некотором специальном смысле), что мы сможем уже доказать этот факт. Может случиться и так, что сама эта операция приведёт к доказательству.\\
Именно: это будет так, если мы сумеем доказать два утверждения: \\
1. $\forall{n}\varphi_n\textrm{ даёт в некотором смысле n-e приближение}\Rightarrow\varphi_{n+1}\textrm{ даёт в некотором смысле (n+1)-e приближение}$ \\
2. При некотором фиксированном $k$ фукнция $\varphi_k$ даёт $k$-e приближение. \\
Простым примером такой ситуации может служить процесс, который привёл нас к выражению \\
\begin{multline}
    x=L_1-L_2+\frac{L_2}{L_1}+\left(\frac{1}{2}L_2^{2}-L_2\right)\frac{1}{L_1^{2}}+\left(-\frac{1}{3}L_2^{3}-\frac{3}{2}L_2^{2}+L_2\right)\frac{1}{L_1^{3}}+\\
    +\left(\frac{1}{4}L_2^{4}-\frac{11}{6}L_2^{3}+3L^{2}+O(L_2)\right)\frac{1}{L_1^{4}}
\end{multline}
Нам повезло, что у нас оказалась полезная информация: $0<x<\ln{t}$, правильная с самого начала, и в догадках необходимости не было.
\newpage
\section{Корни уравнений}
Мы хотим получить приближённое значение для некоторого корня $\xi$ уравнения $f(x)=0$. В этом случае хороший результат даёт метод Ньютона. Он состоит в том, что берется грубое приближение $x_0$
и строится последовательность $x_1, x_2, x_3, \dots$ по формуле \\
$x_{n+1}=x_n-\frac{f(x)}{f^{\prime}(x)}$ \\
Это означает, что $x_{n+1}$ является корнем линейной фукнции график которой-касательная в точке $P_n=(x_n,f(x_n))$ к графику $f(x)$. \\
При этом обычно $\exists$ интервал $J:$ точка $\xi\in{J}$ и если $x_0\in{J}$, $x_1, x_2,\dots\in{J}$ и $\lim_{n\to\infty}x_n=\xi$. \\
Достаточным условием существования $J$ может служить, например, следующее: \\
1. $f(x)$ имеет непрерывную производную в окрестности точки $\xi$ \\
2. $f^{\prime}(\xi)\neq{0}$ \\
При выполнении этих условий процесс сходится очень быстро, а именно $x_{n+1}-\xi$ имеет порядок $(x_n-\xi)^{2}$. \\
Часто о функции $f(x)$ известно очень мало, иначе говоря, для каждого $x$ можно найти значение $f(x)$, но информация относительно верхних и нижних границ $f(x),f^{\prime}(x),\dots$
для больших интервалов оси $x$ не очень велика. Такую информацию обычно можно получить для очень малых интервалов. \\
Чтобы найти корень уравнения $f(x)=0$, мы просто выбираем более или менее случайно некоторое число $x_0$ и строим последовательность $x_1, x_2, \dots$ при помощи итерационного процесса Ньютона. \\
Если эта последовательность обнаруживает тенденцию сходиться, это ещё ничего не означает, поскольку сходимость не может быть установлена с помощью конечного числа наблюдений. \\
Однако может случиться, что рано или поздно мы попадём в малый интервал $J$, в которой информация о $f(x)$ уже достаточно велика, чтобы доказать, что все следующие $x_n$ остаются
в интервале $J$ и сходятся к некоторой точке этого интервала, что эта точка-корень уравнения $f(x)=0$ и что внутри $J$ не существует других корней. \\
Добившись этого, мы будем значть не точное значение корня, а лишь малый интервал, в котором оно заключено; кроме того, мы имеем способ безграничного уменьшения этого интервала. \\
Имеются однако и неблагоприятные возможности, некоторые из них: \\
1. Последовательность $x_0, x_1, \dots$ стремится к бесконечности. \\
2. Последовательность сходится, но не к нужному корню. \\
3. Последовательность колеблится \\
4. Последовательность сходится к нужному корню, но мы не в состоянии это доказать. \\
\newpage
\section{Асимптотические итерации}
Возвращаясь к асимптотическим задачам, связанным с неявными функциями, заметим, что метод Ньютона вполне хорош для задач с малым параметром, подобных тем, которые рассматривались в главе "Применение".
Также корень уже не число, а фукнция от $t$ и нам нужна асимптотическая информация об этой функции. \\
Имеется два различных вопроса: \\
1. Даёт ли метод Ньютона последовательность достаточно хороших приближений. \\
2. Можем ли мы доказать, что эти приближения действительно являеются приближениями. \\
Рассмотрим пример на уравнении $xe^{x}=\frac{1}{t}$. \\
В качестве первого приближения к корню возьмём $\varphi_0=0$. \\
$f^{\prime}(x)=(x+1)e^{x}$
$x_{n+1}=x_n-\frac{x_ne^{x_n}-\frac{1}{t}}{(x_n+1)e^{x_n}}=\frac{x_n^{2}e^{x_n}+x_ne^{x_n}-x_ne^{x_n}-\frac{1}{t}}{(x_n+1)e^{x}}=\frac{x_n^{2}+\frac{1}{te^{x_n}}}{x_n+1}$ \\
и, полагая $\varphi_1=\frac{1}{t}$ получим, что \\
$\varphi_2=\frac{1}{t}-\frac{\frac{1}{t}(e^{\frac{1}{t}}-1)}{e^{\frac{1}{t}}(1+\frac{1}{t})} = \frac{1}{t}-\frac{1}{t^{2}}+\frac{3}{2}\frac{1}{t^{3}}+O\left(\frac{1}{t^{4}}\right)\quad(t\to\infty)$ \\
Перейдём теперь к уравнению $xe^{x}=t$ и применим метод Ньютона на этом этапе, до ввода малого параметра. \\
Разумно начать приближение с $\varphi_0=0$. Имеем \\
$\varphi_1=t$ \\
$\varphi_2=t-1+O(\frac{1}{t})\quad(t\to\infty)$\\
$\varphi_3=t-2+O(\frac{1}{t})\quad(t\to\infty)$ \\
$\dots$ \\
Ясно, что это ни к чему на не приведёт. Ни одна из фукнций $\varphi_k$ совершенно непохожа асимптотически на истинный корень $x=\ln{t}-\ln\ln{t}+o(1)$. \\
То же самое произойдёт, если мы начнём с $\varphi_0=\ln{t}$, мы опять получим, что $\varphi_n=\ln{t}-n+o(1)$. Можно показать, что мы всегда получим $\varphi_n=\varphi_0-n+o(1)$,
если начнём с фукнции $\varphi_0: \frac{\varphi_0e^{\varphi_0}}{t}\to\infty\quad(t\to\infty)$.
Основной целью было подчеркнуть тот факт, что для многих асимптотических задач важно начинать с хорошей гипотезы или с хорошего первого приближения.\\
\newpage
\section{Суммирование}
Мы будем рассматривать суммы вида $\sum_{k=1}^{n}a_k(n)$, где каждое слагаемое и число членов зависят от $n$. Нас будет интересовать асимптотическая информация о значении суммы при
больших значениях $n$. Во многих приложениях $a_k(n)$ не зависят от $n$. \\
Подобные асимптотические задачи моугт быть весьма сложными, особенно в случаях, когда не все $a_k$ одного знака и когда сумма $\sum_{1}^{n}a_k(n)$ может быть много меньше суммы $\sum_{1}^{n}|a_k(n)|$. \\
С другой стороны, имеется класс шаблонных задач, когда все $a_k$ одного знака и "довольно гладкие". Эти задачи мы разделим на четыре типа в зависимости от того, какие слагаемые дают
основной вклад в сумму: \\
1. Сравнительно небольшое число членов в начале или в конце суммы \\
2. один член в начале или в конце \\
3. сравнительно небольшое число членов в середине \\
4. небольшой группы членов, преобладающих на остальными, просто нет \\
\newpage
\section{Случай 1}
В качестве первого примера рассмотрим сумму $s_n=\sum_{k=1}^{n}\frac{1}{k^{3}}$. \\
Первым приближением к $s_n$ является сумма $S=\sum{1}^{\infty}\frac{1}{k^{3}}$ бесконечного ряда, а погрешность равна $-\sum_{n+1}^{\infty}\frac{1}{k^{3}}$. \\
Для этой суммы легко получается оценка $O(\frac{1}{n^{2}})$, например, с помощью такого неравенства: \\
$\sum_{n+1}^{\infty}\frac{1}{k^{3}}<\sum_{n+1}^{\infty}\int_{k-1}^{k}\frac{1}{t^{3}}\,dt=\int_{n}^{\infty}\frac{1}{t^{3}}\,dt=\frac{1}{2n^{2}}$ \\
и, следовательно, \\
$s_n=S+O\left(\frac{1}{n^{2}}\right)\quad(n\to\infty)$ \\
Результаты такого типа вполне удовлетворительны для многих задач анализа, но с точки зрения вычислительной математики это ничего не даёт, если мы не знаем значения $S$. \\
Рассмотрим сумму $s_n=\sum_{1}^{n}2^{k}\ln{k}$. \\
В ней сравнительно небольшое количества последних слагаемых даёт вклад, значительно больший, чем все остальные члены. Если мы отбросим $\left\lfloor\ln{n}\right\rfloor$, то сумма
оставшихся слагаемых не превзойдёт: \\
$\sum_{1}^{n-\lfloor\ln{n}\rfloor}2^{k}\ln{n}\le 2^{n+1-\ln{n}}\ln{n}$ \\
что намного меньше одного только последнего слагаемого. \\
Заметим, что $\ln{k}$ мало меняется, когда $k$ пробегает последние $\lfloor{\ln{n}}\rfloor$ номеров. Поэтому разложим $\ln{k}$ по степеням $\frac{n-k}{n}$; при этом можно считать, что
$\frac{n}{2}<k\le{n}$. Нас удовлетворит оценка \\
$\ln{k}=\ln(n-h)=\ln{n}-\frac{h}{N}+O\left(\frac{h^{2}}{n^{2}}\right)\quad(n\to\infty)$, которая справедлива равномерно по $0\le{h}<\frac{n}{2}$. \\
Теперь проведём следующие оценки: \\
$\sum_{1\le{k}\le\frac{n}{2}}2^{k}\ln{k}=O\left(2^{\frac{n}{2}}\ln{n}\right)$ \\
$\sum_{\frac{n}{2}<k\le{n}}2^{k}\ln{n}=2^{n+1}\ln{n}+O(2^{\frac{n}{2}}\ln{n})$ \\
$\sum_{\frac{n}{2}<k<\le{n}}2^{k}\frac{h}{n}=\frac{2^{n}}{n}\sum_{h=1}^{\infty}2^{-h}h+O\left(2^{\frac{n}{2}}\right)$\\
$\sum_{\frac{n}{2}<k\le{n}}2^{k}O\left(\frac{h^{2}}{n^{2}}\right)=O\left(\frac{2^{n}}{n^{2}}\right)\sum_{h=1}^{\infty}\frac{h^{2}}{2^{h}}$ \\
ДОБАВИТЬ ПОЯСНЕНИЯ \\
Главная часть отстаточного члена равна $O(\frac{2^{n}}{n^{2}})$; члены, содержащие $2^\frac{n}{2}$, значительно меньше. \\
Таким образом, мы получаем \\
$\frac{1}{2^{n}}\sum_{1}^{n}2^{k}\ln{k}=2\ln{n}-\frac{1}{n}\sum_{1}^{\infty}\frac{h}{2^{h}}+O\left(\frac{1}{n^{2}}\right)$ \\
Нетрудно получить асимптотический ряд по степеням $\frac{1}{n}$ \\
$\frac{1}{2^{n}}\sum_{1}^{n}2^{k}\ln{k}-2\ln{n}\approx \frac{c_1}{n}+\frac{c_2}{n^{2}}+\dots\quad(n\to\infty)$, где $c_k=-\frac{1}{k}\sum_{h=1}^{\infty}\frac{h^m}{2^{h}}$ \\
\newpage
\section{Случай 2}
Часто приходится сталкиваться с суммами положительных членов, в которых каждый член имеет по крайней мере тот же порядок, что и сумма всех предыдущих. \\
Рассмотрим пример $s_n=\sum_{k=1}^{n}k!$ \\
Разделив на последний член получим \\
$\frac{s_n}{n!}=1+\frac{1}{n}+\frac{1}{n(n-1)}+\frac{1}{n(n-1)(n-2)}+\cdots+\frac{1}{n!}$ \\
Если мы остановимся, скажем, после пятого члена и пренебрежём последними $(n-5)$, каждый из которых не привосходит $\frac{(n-5)!}{n!}$, то мы сделаем ошибку порядка $O\left(\frac{1}{n^4}\right)$. \\
Но пятый член сам имеет порядок $O\left(\frac{1}{n^{4}}\right)$, так что \\
$\frac{s_n}{n!}=1+\frac{1}{n}+\frac{1}{n(n-1)}+\frac{1}{n(n-1)(n-2)}+O\left(\frac{1}{n^4}\right)\quad(n\to\infty)$ \\
Заменим число 5 произвольным целым числом, мы легко убедимся, что существует асимптотический ряд \\
$\frac{s_n}{n!}\approx c_0+\frac{c_1}{n}+\frac{c_2}{n^2}+\cdots\quad(n\to\infty)$ \\
Ряд $c_0+c_1x+c_2x^2+\dots$ расходится при любом ненулевом $x$. \\
Ряд $c_0+c_1x+c_2x^2+\dots$ возник как формальная сумма степенных рядов для фукнций \\
$1,x,\frac{x^2}{1-x},\frac{x^3}{(1-x)(1-2x)},\frac{x^4}{(1-x)(1-2x)(1-3x)},\dots$ каждый из которых имеет неотрицательные коэффициенты. \\
Таким образом, для любого целого $k$ коэффициенты ряда $c_0+c_1x+c_2x^2+\dots$ больше коэффициентов ряда для фукнции $\frac{x^{k+1}}{(1-x)(1-2x)\dots(1-kx)}$. \\
Последний ряд расходится при $x=\frac{1}{k}$, следовательно, и ряд $c_0+c_1x+c_2x^2+\dots$ расходится при $x=\frac{1}{k}$. \\
Так как $k$ произвольно, радиус сходимости этого ряда равен нулю. \\
ПОПРОБОВАТЬ \\
\newpage
\section{Случай 3}
Типичный пример: \\
$s_n=\sum_{k=1}^{n}a_k(n),\quad a_k(n)=2^{2k}\left(\frac{n!}{k!(n-k)!}\right)^2$ \\
Имеем $\frac{a_{k+1}(n)}{a_k(n)}=\left(\frac{2(n-k)}{k+1}\right)^2$ \\
Следовательно, максимальный член встретится при первом значении $k$, для которого $2(n-k)<k+1$, т.е. около $k=\frac{2n}{3}$. \\
Заметим, в что в этом случае, в отличие от наших прежних примеров, сумма велика по сравнению с максимальным членом. \\
В самом деле, если мы будем изменять $k$ в любом направлении от максимального члена, то $a_k(n)$ менятеся очень медленно при фиксированном $n$. \\
НАРИСОВАТЬ ГРАФИК\\
Другими методами, например с помощью формулы Стирлинга для факториала, иожно показать, что число членов, превосходящих $\frac{1}{2}\max_k{a_k(n)}$ имеет порядок $\sqrt[2]{n}$.\\
Если же $|k-\frac{2n}{3}|$ много больше $\sqrt[2]{n}$, то $a_k(n)$ очень мало по сравнению с максимумом и общая сумма всех таких членов относительно мала. \\
Поэтому наше внимание должно быть сосредоточено на тех $k$, для которых $|k-\frac{2n}{3}|<A\sqrt[2]{n}$. \\
С помощью формулы Стирлинга $a_k(n)$ при таких значениях $k$ можно достаточно хорошо аппроксимировать. \\
\newpage
\section{Случай 4}
В качестве первого примера рассмотрим $a_k(n)=\sqrt[2]{k}$. \\
Имеются два этапа:
\quad 1. Приближение $a_k$ последовательностью $u_k$, для которой сумма $\sum_{1}^{n}u_k$ точно известна, и оно должно быть достаточно хорошим, чтобы обеспечивать сходимость $\sum{1}^{\infty}(a_k-u_k)$ \\
\quad 2. Имеем дело с $\sum_{k=1}^{n}(a_k-u_k)$ \\
Первым приближением к этой сумме служит сумма ряда $S=\sum_{1}^{\infty}(a_k-u_k)$ и мы имеем \\
$s_n=\sum_{1}^{n}a_k=\sum_{1}^{n}u_k+S+\sum_{n+1}^{\infty}(u_k-a_k)$ \\
В последней сумме мы пытаемся приблизить $u_k-a_k$ последовательностью $v_k$, для которой сумма $\sum_{n+1}^{\infty}v_k$ точно известна, а относительно погрешности $\sum_{n+1}^{\infty}(u_k-a_k-v_k)$ известно, что она мала. \\
Этот процесс можно продолжить и дальше. \\
Слабым местом этого процесса является то, что наша инфомрация о значении $S$ очень незначительна. \\
В нашем примере мы можем получить первое приближение к сумме $s_n$ при помощи интеграла \\
$\int_{0}^{n}\sqrt[2]{t}\,dt=\frac{2}{3}n^\frac{3}{2}$ \\
Однако если мы выберем $u_k$ так, чтобы \\
$\sum_{1}^{n}u_k=\frac{2}{3}n^\frac{3}{2}$ \\
этого будет ещё недостаточно. \\
Действительно, ряд с общим членом $k^\frac{1}{2}-\left(\frac{2}{3}k^\frac{3}{2}-\frac{2}{3}(k-1)^\frac{3}{2}\right)$ ещё не будет сходящимся, так как, разлагая $(1-\frac{1}{k})^\frac{3}{2}$ по степеням $\frac{1}{k}$,
мы находим, что верхнее выражение равно $\frac{1}{4}k^{-\frac{1}{2}}+O\left(k^{-\frac{3}{2}}\right)$, а ряд $\sum_{1}^{\infty}k^{-\frac{1}{2}}$ расходится. \\
Но мы опять приблизим частные суммы $\sum_{1}^{n}k^{-\frac{1}{2}}$ интегралом, что даст нам $2\sqrt[2]{n}$. //
Если мы возьмём теперь новые $u_k$, именно $u_k=U_k-U_{k-1},\quad U_k=\frac{2}{3}k^\frac{3}{2}+\frac{1}{2}\sqrt[2]{k}$, \\
мы без труда найдём, что \\
$u_k-a_k=\frac{1}{48\sqrt[2]{k^3}}+O\left(\frac{1}{\sqrt[2]{k^5}}\right)\quad(k\to\infty)$ \\
откуда видно, что ряд $\sum_{1}^{\infty}(u_k-a_k)$ сходится. \\
На втором этапе мы должны приблизить $u_k-a_k$ при помощи $v_k$. Возьмём $v_k=V_{k-1}-V_k$, где \\
$V_k=\frac{1}{24\sqrt[2]{k}},\quad\sum_{n+1}^{\infty}v_k=V_n$ \\
как посказывает нам интеграл \\
$\int_{n}^{\infty}\frac{1}{48\sqrt[2]{k^3}}\,dt=\frac{1}{24\sqrt[2]{n}}$ \\
Таким образом, получим \\
$u_k-a_k-v_k=O\left(k^{-\frac{5}{2}}\right)$ \\
итого получаем \\
$\sum_{1}^{n}k^\frac{1}{2}=\frac{2}{3}n^\frac{3}{2}+\frac{1}{2}n^\frac{1}{2}+S+\frac{1}{24}n^{-\frac{1}{2}}+O\left(n^{-\frac{3}{2}}\right)\quad(n\to\infty)$ \\
Остаточный член $O\left(n^{-\frac{3}{2}}\right)$ можно заменить асимптотическим рядом, поскольку процесс может быть продолжен и мы можем при желании получить любое число членов. \\
Для этого, конечно, необходимо уточнить оценку $u_k-a_k-v_k=O\left(k^{-\frac{5}{2}}\right)$ , что легко сделать, так как $(u_k-a_k)k^\frac{3}{2}$ можно разложить по степеням $\frac{1}{k}$, сходящийся при $k>1$. \\
Остаётся ещё вопрос о значении $S$. Очевидно, имеем \\
$S=\sum_{1}^{\infty}\left(\sqrt[2]{k}-\frac{2}{3}k^\frac{3}{2}-\frac{1}{2}\sqrt[2]{k}+\frac{2}{3}(k-1)^\frac{3}{2}+\frac{1}{2}\sqrt[2]{k-1}\right)=\lim_{n\to\infty}\left(\sum_{1}^{n}\sqrt[]{k}-U_n\right)$ \\
но можно получить и более простое выражение. \\
Этот метод использует аналитичность и поэтому не всегда применим. \\
Сначала обощим выражение $\sum_{1}^{n}k^\frac{1}{2}=\frac{2}{3}n^\frac{3}{2}+\frac{1}{2}n^\frac{1}{2}+S+\frac{1}{24}n^{-\frac{1}{2}}+O\left(n^{-\frac{3}{2}}\right)\quad(n\to\infty)$
, введя комплексный параметр $z$. \\
Тем же способом мы получим, что \\
$\sum_{1}^{n}k^{-z}=\frac{n^{1-z}}{1-z}+\frac{1}{2}n^{-z}+S(z)+O\left(n^{-z-1}\right)\quad(n\to\infty)$ при $Re{z}>-1,z\ne{1}$. \\
Здесь $S(z)$-сумма сходящегося ряда, аналогичного ряду \\
$S=\sum_{1}^{\infty}\left(\sqrt[2]{k}-\frac{2}{3}k^\frac{3}{2}-\frac{1}{2}\sqrt[2]{k}+\frac{2}{3}(k-1)^\frac{3}{2}+\frac{1}{2}\sqrt[2]{k-1}\right)=\lim_{n\to\infty}\left(\sum_{1}^{n}\sqrt[]{k}-U_n\right)$. \\
Кроме того нетрудно показать, что эта сумма является аналитической фукнцией $z$ в области $Re{z}>-1,z\ne{1}$. \\
Если $Re{z}>1$, то она совпадает с дзета-фукнцией Римана $\zeta(z)=\sum_{1}^{\infty}n^{-z}$,  в чём нетрудно убедиться, устремив $n\to\infty$.\\
\newpage
\section{Формула суммирования Эйлера-Маклорена}
При рассмотрении вышенаписанного примера мы использовали метод скорее для демонстрации. Однако, по-видимому, кратчайшим и наиболее эффективным способом исследования в таких случаях является формула Эйлера-Маклорена. \\
Основная формула имеет вид \\
\begin{multline}
\frac{g(0)+g(1)}{2}-\int_{0}^{1}g(x)\,dx=(g^{\prime}(1)-g^{\prime}(0))\frac{B_2}{2!}+(g^{\prime\prime\prime}(1)-g^{\prime\prime\prime}(0))\frac{B_4}{4!}+\dots \\
+(g^{(2m-1)}(1)-g^{(2m-1)}(0))\frac{B_{2m}}{(2m)!}-\int_{0}^{1}g^{(2m)}(x)\frac{B_{2m}(x)}{(2m)!}\,dx
\end{multline}
Здесь $m\geq{1}$-любое целое число, а $g$-фукнция, имеющая $2m$ непрерывных производных на интервале $0\le{x}\le{1}$. \\
Величины $B_k$-числа Бернулли-определяются равенством $\frac{z}{e^z-1}=\sum_{0}^{\infty}\frac{B_n}{n!}z^n\quad(|z|<2\pi)$. \\
Наконец, $B_n(t)$ означает многочлен Бернулли, который определяется равенством $\frac{ze^{zt}}{e^z-1}=\sum_{0}^{\infty}\frac{B_n(t)}{n!}z^n$ \\
Если мы напишем формулу для функции $g(x)=f(x+1),g(x)=f(x+2),/dots,g(x)=f(x+n-1)$ и сложим полученные результаты, то многие слагаемые взаимно уничтожаются, и мы придём к формуле суммирования Эйлера-Маклорена. \\
Запишем её в виде \\
\begin{multline}
f(1)+\cdots+f(n)=\int_{1}^{n}f(x)\,dx+\frac{1}{2}f(n)+\frac{B_2}{2!}f^{\prime}(n)+\frac{B_4}{4!}f^{\prime\prime\prime}(n)+\cdots+\\
+\frac{B_{2m}}{(2m)!}f^{(2m-1)}(n)-\int_{1}^{n}f^{(2m)}(x)\frac{B_{2m}(x-\lfloor{x}\rfloor)}{(2m)!}\,dx
\end{multline}
Функцию $f(x)$ мы будем предполагать имеющей $2m$ непрерывных производных при $x\geq{1}$. Символ $\lfloor{x}\rfloor$ означает наибольшее целое число, не превосходящее $x$,
$B_{2m}(x-\lfloor{x}\rfloor)$-это значение $2m$-го многочлена Бернулли в точке. \\
Число $C$ не зависит от $n$: \\
$C=\frac{1}{2}f(1)-\frac{B_2}{2!}f^{\prime}(1)-\cdots-\frac{B_{2m}}{(2m)!}f^{(2m-1)}(1)$ \\
Известно, что \\
$B_{2m}(x-\lfloor{x}\rfloor)=2(2m)!(2\pi)^{-2m}(-1)^{m+1}\sum_{k=1}^{\infty}k^{-2m}\cos{2k\pi{x}}$ при $m=1,2,3,\dots$, откуда следует, что \\
$|B_{2m}(x-\lfloor{x}\rfloor)|\le |B_{2m}|=2(2m)!(2\pi)^{-2m}\sum_{k=1}^{\infty}k^{-2m}$ \\
Это даёт нам удовлетворительную оценку для остаточного члена для основной формулы в начале главы. \\
Если фукнция $f(x)$ такова, что $\int_{0}^{\infty}|f^{(2m)}(x)|\,dx<\infty$, то мы сразу получаем асимптотическую формулу \\
\begin{multline}
f(1)+\cdots+f(n)=\int_{1}^{n}f(x)\,dx+S+\frac{1}{2}f(n)+\sum_{k=1}^{m}\frac{B_{2k}}{(2k)!}f^{(2k-1)}(n)+ \\ +O\left(\int_{n}^{\infty}|f^{(2m)}(x)|\,dx\right)\quad(n\to\infty)
\end{multline}
где $m\geq{1}$-фиксированное целое число, а $S=C-\int_{1}^{\infty}f^{(2m)}(x)\frac{B_{2m}(x-\lfloor{x}\rfloor)}{(2m)!}\,dx$.
Рассмотрим несколько примеров. \\
1. Пусть $f(x)=x^{-z}\ln{x},\,z\in\mathbb{C}$. \\
Тогда полученная нами выше оценка применима при $2m>1-Re{z}$и даёт нам \\
$\sum_{k=1}^{n}k^{-z}\ln{k}=\int_{1}^{n}x^{-z}\ln{x}\,dx+C(z)+\frac{1}{2}n^{-z}\ln{n}+R(n,z)$, где $C(z)$ зависит только от $z$, а $R(n,z)$ имеет асимптотическое разложение \\
$R(n,z)\approx\frac{B_2}{2!}(n^{-z}\ln{n})^{\prime}+\frac{B_4}{4!}(n^{-z}\ln{n})^{m}+\dots\quad(n\to\infty)$ \\
Здесь берётся производная по $n$, по предположению, что $n$ меняется непрерывно. \\
$C(z)$ можно найти, использовав аналитичность по $z$, и получить, что $C(z)=-\zeta^{\prime}(z)-\frac{1}{(1-z)^{2}}$.\\
Замечание: \\
Грубо говоря, метод Эйлера-Маклорена не приводит к цели, если наибольший член, скажем $f(n)$, не мал по сравнению со всей суммой $f(1)+f(2)+\cdots+f(n)$. \\
В этом случае нельзя ожидать, что порядок $f^{(2m)}(n)$ меньше, чем порядок $f(n)$, и формула Эйлера-Маклорена не может дать ничего лучшего, чем $f(1)+f(2)+\cdots+f(n)=O(f(n))$. \\
Это можно проиллюстрировать на примере $\sum_{1}^{n}k!$. \\
2. Метод Эйлера-Маклорена можно применять и к суммам вида $\sum_{k=1}^{n}a_k(n)$, где каждое слагаемое зависит от $k$ и от $n$. \\
Однако в этом случае не имеет смысле переходить от формулы $(6)$ к $(7)$, потому что тогда $S$ будет зависеть от $n$. \\
Неопределённая постоянная в асимптотической формуле часто вполне допустима, но иметь в такой формуле неопределённую фукнцию от $n$-это значит не иметь никакой формулы вообще.\\
Однако имеются случаи, когда \\
$\frac{1}{(2m)!}\int_{1}^{n}f^{(2m)}(x)B_{2m}(x-\rfloor{x}\lfloor)\,dx$ \\
не доставляет трудностей, например когда интеграл $\int_{1}^{n}|f^{(2m)}(x)|\,dx$ сравнительно мал. \\
В качестве такого примера возьмём \\
$s_n=\sum_{k=-n}^{n}e^{-k^2\frac{@a}{n}}$, где $@a$-положительная постоянная \\
Формула Эйлера-Маклорена при $f(x)=e^{-k^2\frac{@a}{n}}$ даёт \\
$s_n=\int_{-n}^{n}f(x)\,dx+\frac{1}{2}f(n)+\frac{1}{2}f(-n)+\frac{B_2}{2!}(f^{\prime}(n)-f^{\prime}(-n))+\cdots+\frac{B_{2m}}{(2m)!}(f^{(2m-1)}(n)-f^{(2m-1)}(-n)) +R_m$ \\
где $R_m=-\int_{-n}^{n}f^{(2m)}(x)\frac{B_{2m}(x-\rfloor{x}\lfloor)}{(2m)!}\,dx$ \\
откуда $|R_m|\le \frac{|B_{2m}|}{(2m)!}\int_{-n}^{n}|f^{(2m)}(x)|\,dx$ \\
Имеем \\
$\int_{-n}^{n}f(x)\,dx=\int_{-\infty}^{\infty}f(x)\,dx+\varepsilon_n=\sqrt{\frac{\pi\,n}{\alpha}}+\varepsilon_n$, где $\varepsilon_n=O\left(e^{-bn}\right), b>0$. \\
Про такой остаточный член говорят, что он экспоненциально мал. \\
Остальные слагаемые в формуле для $s_n$ тоже экспоненциально малы. Таким образом, все зависит от того, насколько хорошо нам удастся оценить $R_m$. \\
Делая замену $x=y\sqrt{\frac{n}{2\alpha}}$, получаем \\
$\int_{-\infty}^{\infty}|f^{(2m)}(x)|\,dx=\left(\frac{2\alpha}{n}\right)^{m-\frac{1}{2}}\int_{-\infty}^{\infty}\left|\frac{d^{2m}}{dy^{2m}}e^{-\frac{y^2}{2}}\right|\,dy$ \\
и, следовательно, $|R_m|<C_mn^{\frac{1}{2}-m}>0$ и не зависит от $n$. Поэтому при любом $m$ имеем \\ 
$s_n=\sqrt{\frac{\pi\,n}{\alpha}}+O\left(n^{\frac{1}{2}-m}\right)$ \\
Здесь мы случайно можем получить информацию из формулы преобразования $\theta$-фукнции, дающее хорошую оценку для $s_n$. \\
Формула преобазования $\theta$-фукнции даёт \\
$\sum_{k=-\infty}^{\infty}e^{-\alpha\frac{k^2}{n}}=\sqrt{\frac{\pi\,n}{\alpha}}\sum_{k=-\infty}^{\infty}e^{-k^2\pi^2\frac{n}{\alpha}}$ \\ откуда \\
$\int_{-\infty}^{\infty}f^{(2m)}(x)\frac{B_{2m}(x-\rfloor{x}\lfloor)}{(2m)!}\,dx=-2\sqrt{\frac{\pi\,n}{\alpha}}e^{\pi^2\frac{n}{\alpha}}+O\left(n^\frac{1}{2}e^{-4\pi^2\frac{n}{\alpha}}\right)\quad(n\to\infty)$
Если сравнить обе оценки, то видно, что полученная новая оценка лучше при фиксированном $m$, однако мы всё ещё можем методом Эйлера-Маклорена получить оценку $O\left(ne^{-\pi^2\frac{n}{\alpha}}\right)$, 
отличающуюся от истинной на множитель $n^\frac{1}{2}$. \\
Если воспользоваться определением многочленов Эрмита \\
$H_k(y)=(-1)^{k}e^\frac{y^2}{2}\left(\frac{d}{dy}\right)^{k}e^{-\frac{y^2}{2}}$ \\
то подинтегральную фукнцию в правой части, когда мы делали замену, можно записать в виде $e^{-\frac{y^2}{2}}|H_{2m}(y)|$. \\
Используя интегральной представление \\
\begin{multline}
H_{2m}(y)=\frac{1}{\sqrt{2\pi}}e^{\frac{y^2}{2}}\left(\frac{d}{dy}\right)^{2m}\int_{-\infty}^{\infty}e^{-\frac{1}{2}v^2+ivy}\,dv = \\
=\frac{1}{\sqrt{2\pi}}\int_{-\infty}^{\infty}(iv)^{2m}e^{-\frac{1}{2}v^2+ivy+\frac{1}{2}y^2}\,dv = \\
=\frac{1}{\sqrt{2\pi}}\int_{-\infty}^{\infty}(y-iu)^{2m}e^{-\frac{u^2}{2}}\, du
\end{multline}
находим \\
$\int_{-\infty}^{\infty}|f^{(2m)}(x)|dx\le \left(\frac{2\alpha}{n}\right)^{m-\frac{1}{2}}\frac{1}{\sqrt{2\pi}}\int_{-\infty}^{\infty}\int_{-\infty}^{\infty}e^{-\frac{y^2+u^2}{2}}(y^2+u^2)^{m}\,du\,dy$ \\
Вводя полярные координаты в плоскости $(u,y)$ находим, что повторный интеграл равне $2^{m+1}\pi{m!}$. \\
Множитель $\frac{|B_{2m}|}{(2m)!}=\frac{2}{\sqrt[2m]{2\pi}}\zeta(2m)\Rightarrow\frac{|B_{2m}|}{(2m)!}<\frac{C}{\sqrt[2m]{2\pi}}$, где $C$-постоянная. \\
Поэтому получается, что \\
$\int_{-\infty}^{\infty}f^{(2m)}(x)\frac{B_{2m}(x-\rfloor{x}\lfloor)}{(2m)!}\,dx<\frac{C}{\sqrt[2m]{2\pi}}\left(\frac{2\alpha}{n}\right)^{m-\frac{1}{2}}2^{m+1}\pi{m!}$ \\
Используя формулу Стирлинга для факториала, мы заключаем, что \\
$\exists{C_1=const}:\forall{m}\forall{n}\Rightarrow|\int_{-\infty}^{\infty}f^{(2m)}(x)\frac{B_{2m}(x-\rfloor{x}\lfloor)}{(2m)!}\,dx|<C_1\left(\frac{\alpha{m}}{\pi^2ne}\right)^{m}\sqrt{\frac{nm}{2\alpha}}$ \\
Теперь выберем значение $m$. Нетрудно убедиться, что минимум выражения $\min\left(\frac{\alpha{t}}{\pi^2ne}\right)^{t}=e^{-\frac{pi^2n}{\alpha}}$ достигается при $t=\frac{\pi^2n}{\alpha}$. \\
Однако $m$-целое, а потому мы должны взять $m=m_0=\left\lfloor\frac{pi^2n}{\alpha}\right\rfloor$. \\
Чтобы исследовать как изменится при это оценка для интеграла $\int_{-\infty}^{\infty}f^{(2m)}(x)\frac{B_{2m}(x-\rfloor{x}\lfloor)}{(2m)!}\,dx$, положим \\
$\psi(\rho)=\rho\ln\frac{\alpha\rho}{\pi^2e}$ \\
$\min\psi(\rho)=-\frac{\pi^2}{\alpha}$ при $\rho=\rho_0=\frac{pi^2}{\alpha}$. \\
Имеем $\psi^\prime(\rho_0)=0\Rightarrow\psi\left(\frac{m_0}{n}\right)=\psi\left(\rho_0+O\left(\frac{1}{n}\right)\right)=-\frac{\pi^2}{\alpha}+O\left(\frac{1}{n^2}\right)$. \\
Теперь видно, что если $m=m_0$, то мы получим, что \\
$\int_{-\infty}^{\infty}f^{(2m)}(x)\frac{B_{2m}(x-\rfloor{x}\lfloor)}{(2m)!}\,dx=O\left(ne^{-\frac{\pi^2n}{\alpha}}\right)$
\newpage
\section{Формула Стирлинга для $\Gamma$-фукнции в комплексной плоскости}
Наша сумма будет содержать параметр $z$, при фиксированном $z$ мы будем безгранично увеличивать число слагаемых и только после этого заставим $|z|$ стремиться к бесконечности.
Пусть $z$-действительное или комплексное числи такое, что $Re{z}\le0$ и $z\neq{0}$. \\
Применим формулу Эйлера-Маклорена к сумме \\
$S_n(z)=\sum_{k=1}^{n}\ln(z+k-1)$, где для логарифма берётся главное значение. \\
При произвольном целом $m\le{1}$ получаем \\
\begin{multline}
S_n(z)=\frac{1}{2}\ln{z}+\frac{1}{2}\ln(z+n-1)+\int_{1}^{n}\ln(z+x-1)\,dx+ \\
+\sum_{k=1}^{m}((z+n-1)^{1-2k}-z^{1-2k})\frac{B_{2k}}{2k(2k-1)}+\int_{1}^{n}(z+x-1)^{-2m}\frac{B_{2m}(x-\rfloor{x}\lfloor)}{2m}\,dx
\end{multline}
При фиксированном $z$ мы без труда получаем асимптотическую формулу с остаточным членом $o(1)$:
$S_n(z)=\left(z-\frac{1}{2}\right)\ln{n}-\left(z-\frac{1}{2}\right)\ln{z}+n\ln{n}+z-n-\rho(z)+o(1)\quad(n\to\infty)$ \\
где \\
$\rho(z)=\sum_{k=1}^{m}z^{1-2k}\frac{B_{2k}}{2k(2k-1)}-\int_{0}^{\infty}(z+x)^{-2m}\frac{B_{2m}(x-\rfloor{x}\lfloor)}{2m}\,dx$ \\
Поскольку фукнция $\rho(z)$ не зависит от $n$. Интегрируя по частям в правой части равенства, можно убедиться, что $\rho(z)$ не зависит и от $m$. \\
Используя полученные формулы, получим \\
\begin{multline}
S_n(z)-S_n(1)=(z-1)\ln{n}-\left(z-\frac{1}{2}\right)\ln{z}+z-1+\rho(1)-\rho(z)+o(1)\quad(n\to\infty)
\end{multline} 
Эта разность связана с формулой Эйлера для $\Gamma$(z): \\
$\Gamma(z)=\lim_{n\to\infty}\frac{n^{z-1}n!}{z(z+1)(z+2)\dots(z+n-1)}$
логарифмируем \\
$\ln\Gamma(z)=\lim_{n\to\infty}((z-1)\ln{n}+S_n(1)-S_n(z))$ \\
$\ln\Gamma(z)=\left(z-\frac{1}{2}\right)\ln{z}-z+\rho(z)+1-\rho(1)$ \\
Следует заметить, что $\ln\Gamma(z)$  не обязательно является главным значением логарифма. \\
Из этого тождества уже нетрудно вывести асимптотическую формулу при $|z|\to\infty$. \\
Пусть $\delta=const: 0<\delta<\pi,\,R_\delta={z| |\arg{z}|<\pi-\delta},\,m\in\mathbb{Z}\geq{1}$. \\
Тогда $B_{2m}(x-\rfloor{x}\lfloor)$ ограничено, откуда следует, что \\
\begin{multline}
\int_{0}^{\infty}\left|(z+x)^{-2m}\frac{B_{2m}(x-\rfloor{x}\lfloor)}{2m}\right|\,dx<C\int_{0}^{\infty}|z+x|^{-2m}\,dx=\\
=C|z|^{-2m+1}\int_{0}^{\infty}\left|y+\frac{z}{|z|}\right|^{-2m}\,dy
\end{multline}
причём $C$ не зависит от z. \\
Величина $\left|y+\frac{z}{|z|}\right|$ равна расстоянию от точки $-y$ до некоторой точки единичного круга, принадлежащей области $R_\delta$. \\
НАРИСОВАТЬ \\
Из геометрических соображений ясно, что расстояние не меньше, чем $\left|y+e^{i(\pi-\delta)}\right|$. \\
Так как интеграл $\int_{0}^{\infty}\left|y+e^{i(\pi-\delta)}\right|^{-2m}\,dy$ сходится, то \\
$\int_{0}^{\infty}(z+x)^{-2m}\frac{B_{2m}(x-\rfloor{x}\lfloor)}{2m}\,dx=O(|z|^{1-2m})$ \\
Поскольку $m$ произвольно, мы получаем асимптотический ряд для $\rho(z)$. Получаем \\
\begin{multline}
\ln\Gamma(z)-\left(z-\frac{1}{2}\right)\ln{z}+z\approx \\
\approx 1-\rho(1)+\sum_{k=1}^{\infty}z^{1-2k}\frac{B_{2k}}{2k(2k-1)}\quad(|\arg{z}|<\pi-\delta, |z|\to\infty)
\end{multline}
Можно показать, что $1-\rho(1)=\frac{1}{2}\ln{2\pi}$.
\newpage
\section{Знакопеременные суммы}
Знакопеременная сумам-это сумма вида $\sum(-1)^{k}f(k)$, где $f(k)\geq{0}$. Естественно ожидать, что такие суммы малы, т.е. намного меньше, чем сумма абсолютных величин слагаемых. \\
Мы можем, конечно, написать \\
$\sum_{k=0}^{2m+1}(-1)^{k}f(k)=\sum_{0}^{m}f(2k)-\sum_{k=0}^{2m}f(2k+1)$ \\
и исследовать каждую сумму в отдельности, но обычно эти суммы почти равны, так что приходится проводить оценки с большей точностью, чтобы получить о разности достаточную информацию. \\
В большинстве случаев проще сгруппировать слагаемые попарно: \\
$\sum_{k=0}^{2m+1}(-1)^{k}f(k)=\sum_{k=0}^{m}(f(2k)-f(2k+1))$, $f(2k)-f(2k+1)$ обычно малы. \\
В качестве примера рассмотрим $S(t)=\sum_{k=0}^{\infty}(-1)^{k}\frac{1}{\sqrt{k^2+t^2}}$ и исследуем асимптотическое поведение $S(t)$ при $t\to\infty$. \\
Функция $f(x)=\frac{1}{\sqrt{x^2+t^2}}$ убывает и стремится к нулю при $x\to\infty$. Поэтому ряд сходится и $0<S(t)<f(0)$. \\
Таким образом, первая грубая оценка $S(t)=O\left(\frac{1}{t}\right)$ \\
Дальше представим $f(2k)-f(2k+1)=-\frac{1}{2}\int_{2k}^{2k+2}f^\prime(x)\,dx$
\end{document}